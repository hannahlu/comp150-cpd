% Please do not change the document class
\documentclass{scrartcl}

% Please do not change these packages
\usepackage[hidelinks]{hyperref}
\usepackage[none]{hyphenat}
\usepackage{setspace}
\doublespace

% You may add additional packages here
\usepackage{amsmath}

% Please include a clear, concise, and descriptive title
\title{Continued Professional Development}

% Please do not change the subtitle
\subtitle{COMP160 - CPD Report}

% Please put your student number in the author field
\author{1608351}

\begin{document}

\maketitle

\section{Introduction}

As discussed in my previous CPD, I am interested in the development of accessible games for children with special educational needs. During the semester I have explored accessile gaming in both the software engineering essay and in researching accessible gaming controllers for the comp140 hardware project. Researching for the essay, has strenghtenned my understanding of  software accessibility requirments, and how these can be implemented using exisiting technologes and models. Software engineering practices, such as the use of onotlogy and traceability, are important in designing robust and accessible software. This semester, I have struggled to understand and implement such practices, and therefore, have included software engineering practices in the five key areas I wish to improve. The further areas I have identified are;  programming language proficiency,  balancing assignments alongside other responsibilities, making better use of opportunities and improving concentration. 
\cite{shannon}

\section{Programming Language Proficiency}

During this semester, I have increased my time dedicated to programming practice. However, this has been heavily focused on blueprints rather than C++. Due to poor time management, my programming has been driven by the need to get mechanics in game as fast as possible. As such, I have not planned the architecture of my code effectively, leading to poor maintainability.

In addition, I have neglected practicing C++. As proficiency in more than one programming languages, and especially C++, allows for more opportunity upon graduation, I wish to focus on C++ next semester, using blueprints predominantly for functions designers and artists need access to.

Over summer I would like to build a small game in Unreal using C++, focusing on best practices for the entire developmental process, from ascertaining requirements and designing architecture through to implementation and testing. As I will not have the same deadline pressures as during term time, I hope to be able to practice dedicating the required time to each stage of development. In the games industry, it is important not to undervalue best practices in software engineering, as it can prevent issues later, produce more maintainable code and help prioritize features needed.

\section{Software Engineering Practices}

In our Comp160 module, I have struggled to understand how to use ontology, such as UML diagrams, in planning a software's architecture. In my research, there is evidence of a tendency for students to undervalue the need for software engineering best practices [][]. In our lectures on UML diagrams and databases, I found we tended to diminish the need for the tasks assigned to us. This may have been due to time pressures related to our Comp160 and Comp140 games, leading us to see more theory based lectures as time taken from the practical work we needed to complete. Due to my lack of focus in these classes I am unconfident in this area and aim to use my summer project to practice implementing software engineering practices. 

Next semester, I aim to start taking notes during lectures as I think this will help improve my concentration and understanding of topics being discussed. I also have a Dictaphone which may be worth experimenting with, although I am unsure how effective I will find this.

\section{Balancing Different Assignments and Other Responsibilities}

The ability to balance workloads as well as outside responsibilities, is an important skill for a future career in the games industry. Using my time at university to practice meeting deadlines, balancing my time between different assignments and preventing outside responsibilties detracting from my studies, will help prepare me for the increased pressures of the workplace. I have found the social pressures of our group project has led me to focus heavily on the game, therefore, falling behind on other assignments. 

Although, I have plotted a study timetable at the start of each week, I have struggled to follow these. For next semester, I plan to assign each week day to a particular assignment to help ensure I spend adequate time on all assignments, without the ridigity of a timetable.

\section{Making Use of Opportunities}

During our first year, I have not made use of the opportunities available to us, such as; Global Game Jams, guest lectures and volunteering at events. I have found adjusting to the amount of social time needed at university challenging, especially at the start of the year, making attending extra events difficult. However, in our second year, I would like to make better use of such opportunities. As I do struggle socialising for extended periods, I aim to commit to one activity a month to start with and increase as suits. The Global Game Jams will be especially beneficial to me, giving me a short and focused time-frame to practice programming and team skills, as well as the chance to work with and learn from the years above us.


\section{Concentration}

Concentration is an area I identified previously, which needs to improve. Maintaining concentration is something I find difficult in general, for example; I struggle to sit and watch films or even one hour TV episodes from start to finish. 


\section{Conclusion}

Write your conclusion here. Though the conclusion should be brief, no more than 100 words, it should do more than merely summarise the report. Focus on the five SMART actions that you intend to take in order to overcome any challenges and/or obstacles. Contextualise how this will help you towards your intended career goal and how this may improve your project for the next semester.

\bibliographystyle{ieeetran}
\bibliography{references}

\end{document}
